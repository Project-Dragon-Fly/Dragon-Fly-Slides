%%%%%%%%%%%%%%%%%%%%%%%%%%%%%%%%%%%%%%%%%
% Beamer Presentation
% LaTeX Template
% Version 1.0 (10/11/12)
%
% This template has been downloaded from:
% http://www.LaTeXTemplates.com
%
% License:
% CC BY-NC-SA 3.0 (http://creativecommons.org/licenses/by-nc-sa/3.0/)
%
%%%%%%%%%%%%%%%%%%%%%%%%%%%%%%%%%%%%%%%%%

%----------------------------------------------------------------------------------------
%	PACKAGES AND THEMES
%----------------------------------------------------------------------------------------

\documentclass{beamer}

\mode<presentation> {
	\usetheme{Madrid}

	%\setbeamertemplate{footline} % To remove the footer line in all slides uncomment this line
	%\setbeamertemplate{footline}[page number] % To replace the footer line in all slides with a simple slide count uncomment this line
	
	%\setbeamertemplate{navigation symbols}{} % To remove the navigation symbols from the bottom of all slides uncomment this line
	
	%\addtobeamertemplate{frametitle}{}{\section{\insertframetitle}} % To add all frame title into table of contents automacitally
	
	%\setbeamertemplate{items}[square]
}

\usepackage{graphicx}
\usepackage{booktabs} % Allows the use of \toprule, \midrule and \bottomrule in tables


%----------------------------------------------------------------------------------------
%	TITLE PAGE
%----------------------------------------------------------------------------------------

\title[DragonFly]{Project DragonFly \\ Surveil Transport sector into Smart City}
\author[Group 3]{
	{\small \textit{Guided by:}} Dr. Rafeeque P.C \\
	\medskip
	{\small \textbf{\textit{Group 3}}} \\
	Abhinand C \\
	Edwin Jose George \\
	Lavanya E.V \\
	Shilpa Suresh
}
\institute[GCEK]{Government College of Engineering Kannur}
\date{\today}

\begin{document}

\begin{frame}
\titlepage
\end{frame}

\begin{frame}
\frametitle{Contents}
\tableofcontents
\end{frame}

%########################################################################################
%	PRESENTATION SLIDES
%########################################################################################

\section{Introduction}
\begin{frame}{Introduction}
    Policing agencies have setup vast networks of distributed surveillance cameras along major routes. The sheer amount of data generated is overwhelming for manual analysis, to trace routes of rouge vehicles or optimally manage traffic, and in practicality the surveillance cameras serves just as an after event mechanism to verify. Significant amount of human resource would be required to utilize cameras as intended and even to alert accidents in real time. \\~\\

    The wide gap in human resource availability and the tedious effort required for traffic analysis, results in under utilization of surveillance infrastructure for policing and improper traffic management.    
\end{frame}


%########################################################################################

\section{Motivation}
\begin{frame}{Motivation}
	\begin{enumerate}
	    \item Public APIs for location of public transport, and tracking system (like Indian Railway NTES system) enlighten public with current traffic and its trends.
	    \item Provides facility to decide when, where, and which transport modes are available.
		\item Enable tracking of vehicles, ie, retrieving route history and related info by typing in the description helps authority in promoting quality flow of traffic.
		\item Aids policing agencies to track routes of rouge vehicles and automate surveillance footage analysis.
		\item Aids in better Alert Control System, \& emergency services in case of accidents and/or other events.		
	\end{enumerate}
\end{frame}

%########################################################################################

\section{Literature Review}
\begin{frame}[allowframebreaks]{Literature Review}
	Trends in Vehicle Re-Identification Past, present and future \cite{deng2021trends}

	CityFlow: A City-Scale Benchmark for Multi-Target
Multi-Camera Vehicle Tracking and Re-Identification \cite{Tang_2019_CVPR}

Urban traffic optimization using traffic cameras as sensors is driving the need to advance state-of-the-art multitarget multi-camera (MTMC) tracking. 

To achieve this goal, one has to address three distinct but closely related research problems: 1) Detection and tracking of targets within a single camera, known as multi-target singlecamera (MTSC) tracking; 2) Re-identification of targets
across multiple cameras, known as ReID; and 3) Detection
and tracking of targets across a network of cameras, known
as multi-target multi-camera (MTMC) tracking. 


The two main challenges in vehicle
ReID are small inter-class variability and large intra-class
variability, i.e., the variety of shapes from different viewing
angles.

A major limitation of existing benchmarks for object
ReID is the limited spatial coverage and small number of cameras used.


Prev literature
	the cameras span less than 300 × 300 m2 ,
with only 6 and 8 views, respectively. 
	Do not provide the original videos or
camera calibration information. 
	Assumes that image signatures
are grouped by correct identities within each camera, which
is not reflective of real tracking systems. 
	Only the front and back views of the
vehicles are available, thus limiting the variability due to
viewpoint. 
	
	Most state-of-the-art approaches on these
benchmarks exploit metric learning to classify object identities, where common loss functions include hard triplet
loss [13], cross entropy loss [40], center loss [48], etc.
	
	On the other hand, the computation of deep learning features is costly, and thus spatio-temporal reasoning using
video-level information is key to applications in the real
world. 
	

In this paper
The first
benchmark at city scale for MTMC tracking
the nature of the synchronized high-quality videos,
the large spatial expanse captured by the
dataset. 

contains the largest number of cameras (40) from a large
number of intersections (10) 
covering a variety of scenes 
fully labeled, homography matrices that relate pixel
locations to GPS coordinates are available to enable precise spatial localization. 


It is important to leverage the spatio-temporal
information to address the city-scale problem properly.



For the person ReID problem, the state-of-the-art apply
metric learning with different loss functions, such as hard
triplet loss (Htri) [13], cross entropy loss (Xent) [40], center loss (Cent) [48], and their combination to train classifiers [62]. 

In our experiments, we compared the performance of various convolutional neural network (CNN) models [12, 54, 16, 51, 17, 38, 36], which are all trained
using the same learning rate (3e-4), number of epochs (60),
batch size (32), and optimizer (Adam). 

For the vehicle ReID problem, the recent work [18] ex-
plores the advances in batch-based sampling for triplet em-
bedding that are used for state-of-the-art in person ReID
solutions. They compared different sampling variants and
demonstrated state-of-the-art results on all vehicle ReID
benchmarks [28, 26, 52], outperforming multi-view-based
embedding and most spatio-temporal regularizations (see
Tab. 7). Chosen sampling variants include batch all (BA),
batch hard (BH), batch sample (BS) and batch weighted
(BW), adopted from [13, 35]. The implementation uses
MobileNetV1 [15] as the backbone neural network architec-
ture, setting the feature vector dimension to 128, the learn-
ing rate to 3e-4, and the batch size to 18 × 4.




5.2. MTSC tracking and object detection
Reliable cross-camera tracking is built upon accurate
tracking within each camera (MTSC). 

As for MTSC trackers, TC [43], the
only offline method, performs better according to most of the evaluation metrics. 


MTMC tracking is a joint process of visual-spatio-temporal reasoning. 
For these experiments, we first apply
MTSC tracking, then sample a number of signatures from
each trajectory in order to extract and compare appearance
 features. The number of sampled instances from each vehi-
cle is empirically chosen as 3. 

Note also that,
since only trajectories spanning multiple cameras are in-
cluded in the evaluation, different from MTSC tracking,
false positives are considered in the calculation of MTMC
tracking accuracy.


We can also conclude from Tab 9 that the
choice of image-based ReID and MTSC tracking methods
has a significant impact on overall performance, as those
methods achieving superior performance in their sub-tasks
also contribute to higher MTMC tracking accuracy.


We proposed a city-scale benchmark, CityFlow, which
enables both video-based MTMC tracking and image-based
ReID tasks. 

Our major contribution is three-fold.
	CityFlow is the first attempt towards city-scale applications
in traffic understanding. 
	
	CityFlow is also the first benchmark to
support vehicle-based MTMC tracking, by providing annotations for the original videos, the camera geometry, and
calibration information. The provided spatio-temporal information can be leveraged to resolve ambiguity in image based ReID.
	
	Third, we conducted extensive experiments
evaluating the performance of state-of-the-art approaches
on our benchmark, comparing and analyzing various visual-spatio-temporal association schemes. 
	
	\include{Real-Time Vehicle Make and Model
Recognition System}
	Efficient and Deep Vehicle Re-Identification Using
Multi-Level Feature Extraction

First we shortlist the vehicle from a gallery set on the
basis of  appearance, and then in the second step we verify the shortlisted vehicle’s license plates with
a query image to identify the targeted vehicle. In our model, the global channel extracts the feature
vector from the whole vehicle image, and the local region channel extracts more discriminative and
 salient features from different regions. In addition to this, we jointly incorporate attributes like model,
type, and color, etc.


Hence, extracted visual
features should be more discriminative and capable of representing some salient identification marks,
such as decoration stickers, annual inspection marks, etc. [12].

The multi-branch network
consisted of two different blocks to learn labels: (1) A backbone network responsible for extracting
appearance features consisting of a dense block. (2) A two-branch network responsible for extracting
model and color features consisting of a coffee-net and inception block, respectively.



\end{frame}


%########################################################################################

\section{Problem Statement}
\begin{frame}{Problem Statement}
	A novel approach to provide a highly integrated platform utilizing existing infrastructure to monitor public transport system, and extend timely information to both public and policing agencies, with intuitive interface.
\end{frame}

%########################################################################################

\section{Objective}
\begin{frame}{Objective}
\end{frame}

%########################################################################################

\section{Contributions Made}
\begin{frame}{Contributions Made}
\end{frame}

%########################################################################################

\section{Requirements}
\subsection{Functional Requirements}
\begin{frame}{Functional Requirements}
\end{frame}

\subsection{Non-functional Requirements}
\begin{frame}{Non-functional Requirements}
    \begin{itemize}
        \item To be used by non technical public - appealing and simple interface
        \item Provide reliable and consistent information and service
        \item Higher concerns on security of personal and public data.
        \item Must account for mobility and agility.
    \end{itemize}
    
    \begin{block}{Execution qualities}
        Good interpretation of data which results reasonable processing of information, considering security of processed data.
    \end{block}

    \begin{block}{Evolution qualities}
        Large volume of data needs to be interpreted and processed as clusters. Needs to account for scalability of resources and services, along with extension of new services.
    \end{block}
\end{frame}

%########################################################################################

\section{Proposed Framework}
\begin{frame}{Proposed Framework}
\end{frame}

%########################################################################################

\subsection{General Architecture/Methodology/ Frame Work/ Technology}
\begin{frame}{Architecture}
\end{frame}

\subsection{Detailed Design}
\begin{frame}{Detailed Design}
\end{frame}

\subsection{Dataset/Input}
\begin{frame}{Dataset}
\end{frame}

%########################################################################################

\section{Work done so far}
\begin{frame}{Work done so far}
\end{frame}

%########################################################################################

\section{Work planned}
\begin{frame}{Work Planned}
\end{frame}

%########################################################################################

\section{Conclusion}
\begin{frame}{Conclusion}
\end{frame}

%########################################################################################

\section{References}
\begin{frame}[allowframebreaks, noframenumbering]{References}
	\nocite{*}
	\bibliographystyle{IEEEtran}
	\bibliography{IEEEabrv,reference.bib}
\end{frame}

%########################################################################################

\end{document} 