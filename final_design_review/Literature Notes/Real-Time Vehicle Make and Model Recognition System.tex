Real-Time Vehicle Make and Model
Recognition System \cite{manzoor2019real}

A Vehicle Make and Model Recognition (VMMR) 

Few of the challenges are image acquisition, variations in illuminations and weather, occlusions, shadows,
reflections, large variety of vehicles, inter-class and intra-class similarities, addition/deletion of vehicles’
 models over time, etc. 


We extract image
features from vehicle images and create feature vectors to represent the dataset. 
We use two classification
algorithms, Random Forest (RF) and Support Vector Machine (SVM), 

The proposed VMMR system recognizes vehicles on the basis of make, model, and
generation (manufacturing years) while the existing VMMR systems can only identify the make and
model.

Comparison with existing VMMR research demonstrates superior performance of the proposed
system in terms of recognition accuracy and processing speed.


Computer vision techniques are used to express images in fewer attributes that characterize vehicles.
Machine learning techniques are used to classify the vehicles. 

The global
feature representation module may generate a model for the encoding of images’ features depending on
the applied technique.
The classifier module produces a model as a result of the training process
that is then used by the testing module to predict the outcome for the newly examined images. 

The proposed system works without the global feature representation
component. 

The extracted features are directly fed into the classifier. The omission of the global feature
representation component improves the processing speed of the VMMR without degrading the recognition
 accuracy.


We define the ROI to represent the part of the vehicle
in the image that provides the discriminative and prominent features. The discriminative and prominent
features are easily distinguishable between different vehicles. We have the frontal vehicle images in our
work to design the VMMR and used bumper, front lights, and bonnet area as ROI


As we increase the
number of blocks in HOG the computational time increases.

The VMMR system performance depends on mainly two elements. The first element is the
representation of dataset which includes the feature extraction techniques and global feature representation
techniques. The second element is the machine learning classification algorithm. Both feature
representation and machine learning algorithm affect the VMMR system. 

Observing the accuracies
of SVM-VMMR and RF-VMMR, we can see that the SVM-VMMR performs better than RF-VMMR. 


A
real-time VMMR system installed on a four-lane highway, covering both sides, is required to process six
vehicles per second approximately in order to analyze entire traffic flow



This work presents a real-time VMMR system with better performance than existing VMMR systems
in terms of recognition rate and processing speed.

The proposed system works well in challenging
situations where vehicles are partially occluded, partially out of the image frame or poorly visible due to
low lighting. 

This system can provide great value in terms of vehicle monitoring and identification based
on vehicle appearance instead of the vehicles’ attached license plate. 

The existing VMMR research focuses
on recognizing vehicles sufficiently to report only their make and model. We have included generation as
another parameter. 

Although the proposed VMMR system outperforms the previous systems, it can be further enhanced.
Image feature vectors have a large number of features/dimensions. 

Dimensionality reduction techniques
can be explored to reduce this number. A publicly available better and larger dataset with more vehicle
types will benefit the research in this area. Deep learning techniques can also be explored with a bigger
dataset.