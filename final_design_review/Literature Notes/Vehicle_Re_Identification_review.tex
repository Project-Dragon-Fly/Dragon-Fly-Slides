\textbf{Trends in Vehicle Re-Identification Past, present and future \cite{deng2021trends}}

Vehicle Re-identification (re-id) over surveillance camera network with non-overlapping
field of view is an exciting and challenging task in intelligent transportation systems (ITS). 

it becomes
more difficult due to inter-class similarity, intra-class variability, viewpoint changes, and spatio-temporal uncertainty.

Vision-based vehicle
re-id approaches, including vehicle appearance, license plate, and spatio-temporal characteristics. 


re-identify the specific
vehicle that appeared in different cameras over the surveillance network. The vehicle re-id
module should recognize same vehicle that appears in surveillance cameras installed
in different geographical locations. Vehicle re-id can be treated as a fine-grained
recognition problem that identifies the subordinate type of input class.

Vehicle re-id can be of six categories
\begin{enumerate}
	\item Vision based Re-Id
	\item Magnetic Sensor Based Re-Id
	\item Inductance Based Re-Id
	\item GPS Based Re-Id
	\item Contextual Cues Based Re-Id
	\item Hybrid Methods Based Re-Id
\end{enumerate}


In computer vision, the aim of vehicle re-id is to identify specific vehicle that appeared
over in multiple cameras network. 
Five Basic steps
\begin{enumerate}
\item Data Collection
\item Bounding Box Generation obtained by vehicle detection technique.
\item Training Data Annotation
\item Model Training:  Learning discriminative
features and good values for all the weights and the bias from previous annotated vehicle videos or images of the dataset.
\item Vehicle Retrieval
\end{enumerate}

Eight different techniques have been employed in this research
area: 
(A) Feature representation for vehicle re-id, 

	Classified into two parts: hand-crafted and deep learning features representations. Handcrafted feature representation utilized in person
	re-id and then applied directly on vehicle re-id task. Deep learning based feature representations such as GoogLeNet [38], VGGNet [39], AlexNet [40], and,
ResNet [41] are used for vehicle re-id.
	
	Deep Joint Discriminative Learning (DJDL) [45] approach uses identification, and verification and
 triplet loss functions improved triplet convolutional neural network [46] uses classification and-oriented and triplet loss function to extract discriminative feature representation.
	
	
(B) Similarity metric for vehicle re-id,
	Distance metric learning approaches [56]
are thoroughly studied in image retrieval and recognition tasks, in which metric space is
	defined in such a way that features that belong to same class are kept closer and different are at distant
	
	As in various face recognition algorithms [57,58] uses Euclidean and
	Cosine distance metric to measure the similarity between the pair of vehicle for re-id.
	
	Furthermore,
deep relative distance learning (DRDL) [44] studied a two-branch convolutional neural
network to covert the raw vehicle images into a Euclidean space, so that distance can be
used directly to measure the similarity of two individual vehicles.
	

(C) Traditional machine learning-based vehicle re-id, 
	Extracted features are directly computed from image pixels and its low level feature representation.
	
	Algorithms proposed for low level feature extraction
		Speeded up robust features (SURF) [47],
		scale-invariant feature transform (SIFT) [48], 
		histogram of oriented gradient (HOG).
		
	After feature extraction different classifiers are applied, which are widely used in TML
approaches such as linear regression, k-Nearest Neighbor (KNN) [49], logistic regression,
support vector machine (SVM) [50], bayes classification [51], and decision tree [52]. 
	
	Zapletal and
Herout [53] utilize the color histogram and the HOG features with linear regression to
re-id vehicle. 
	Chen et al. [54] designed a method to re-id vehicles grid-by-grid with HOG
features extraction for coarse search and further improves the result by utilizing histograms
of matching pairs.
	
	
(D) View-aware-based vehicle re-id,

	Most of the above discussed deep learning features [38,39,45] are general, and these
learned features end at multiple fully connected layers. 
	
	Despite that, all these approaches
performance is not bad. But these approaches are not designed for a specific problem
related to view point variation. 
	
	Zhao et al. [64]
designed a novel approach based
on person body parts guided for re-id. Wu et al. [65] proposed a study with pose prior
	that made identification efficient and robust to viewpoint. Zheng et al. [66] proposed
the pose box structure that generates the pose estimation after affine transformations. Prokaj et al. [68] proposed a pose estimation-based approach to handle multiple
viewpoint problem.
	
	Yi Zhou et al. [69] studied uncertainty in the viewpoint of vehicle re-id
system and designed end to end deep learning-based architecture on Long Short-Term
Memory (LSTM) bi-directional loop and concatenated CNN, in this model author takes
full advantage of LSTM and CNN to learn the different viewpoints of vehicle. 
	
(E) Fine-grained visual recognition-based vehicle re-id, 
	Divided into two parts, representation learning model and part-based model. 
	
	Many
approaches are proposed [60] that utilize alignment and part localization for feature extraction of main parts and then those parts are compared for vehicle re-id. 
	
		Reinforcement learning to get
discriminative parts of vehicle.
		Bilinear architecture
to get the pair of local features
		Utilize shape and lights of vehicle visible in night 
		
		
	In fine-grained recognition, local region features are extracted from different points
such as logo, annual inspection stickers, and decorations, to make system more efficient
and robust various attributes of vehicles are also incorporated like color, model, and type
information. 
	
(F) Generative adversarial network-based vehicle re-id, 
	GAN [71] is one of the hot technique in semi-supervised and unsupervised learning
algorithms. It is proposed by deriving backpropagation signals through
a competitive process involving a pair of networks. 
	
	Generative Adversarial Network (GAN) in Object re-id is among the
	latest research trends in the deep learning approaches. 
	
	Zhou et al. [78] proposed GAN based model to solve cross-view vehicle re-id problem by
generating vehicle images in different viewpoints. Lou et al. [74] designed a model to
generate the same and cross-view vehicle images from original images to facilitate training model. 
	
(G) Attention mechanism, 
	Researchers are trying hard to design an efficient attention-based neural network for
vision-related applications. Such as image classification [80], fine-grained image recognition [81], action recognition [82], and re-id [83]. The commonly followed strategy in these
approaches is integrating a hard part selection subnet work or soft mask branch into the deep networks. 
	
	Zhao et al. [84] studied the part-localization CNN for predicting
	salient parts and features of these parts exploit for person re-id. Wang et al. [80] utilizes
residual learning technique [41] to develop the Residual Attention unit for soft mask learning and gained significant image classification results. 
	
(H) License plate-based vehicle re-id.

	Simply the system’s ability to automatically detect, extract, and recognize license plate characters automatically from vehicle image.
	An automatic License plate recognition (LPR) system is mainly divided into two parts, first license plate detection and
second, interpreting the vehicle license plate image into numerically readable form. 
	
	Li and Shen [91] studied a sequence labelling-based approach using recurrent neural networks (RNN). Super-resolution is also proposed to
	restore a license plate image to improve performance. Shi et al. [92] designed convolutional recurrent neural network (CRNN) for scene text recognition that incorporates feature
extraction, transcription and sequence modeling into a unified framework. 



\textit{\textbf{Hybrid Methods-Based Vehicle Re-Identification
}}
Researchers
have proposed the approaches in which they combined the two or more different techniques.

Liu et al. [42] proposed a framework with name PROVID, in this
framework author not only consider the visual appearance of vehicle for re-id system, but
also exploits the license plate and spatio-temporal cues of vehicle 

Jiang et al. [98] studied vehicle re-id algorithm using appearance and contextual information, author examines the multiple attributes during training like vehicle model, color, and
vehicle image features individual respectively and sort vehicles on the bases of spatio-temporal cues. 

Shen et al. [97] designed a two-step architecture, a pair of query vehicle
images with contextual information and visual temporal path are produced using Markov
Random Fields (MRF) chain model, and then the similarity score is generated.






\textbf{\textit{Evaluation}}
To measure the performance of vehicle re-id approaches, the
cumulative matching characteristics (CMC), curve HIT@1 and HIT@5 are commonly used
by researchers. CMC curve provides the probability that an input image identity appears in a different-sized gallery set. The cumulative number of correctly
matched inputs is demonstrated based on the rank list in which inputs are re-identified.
Moreover, HIT@1 is precision at rank-1 and HIT@5 is precision at rank-5. Rank is utilized to
measure the matching score of test image to its own class, and higher value of rank indicates
the improved performance of the system. 


Besides, CMC curves, if
multiple ground truths for each query image in the gallery set are available, mean average
precision (mAP) is used to measure the overall performance for vehicle re-id system. The mAP measures the overall performance of vehicle re-id system.

Another way by which vehicle re-id techniques performance can be evaluated is the
confusion matrix. A confusion matrix consists of various columns and rows; it depends on
the number of classes. It’s diagonal represents the recognizing accuracy or true classification
and off-diagonal express the misclassification.