\textbf{Trends in Vehicle Re-Identification Past, present and future \cite{deng2021trends}}

Vehicle Re-identification (re-id) over surveillance camera network with non-overlapping
field of view is an exciting and challenging task in intelligent transportation systems (ITS). 

it becomes
more difficult due to inter-class similarity, intra-class variability, viewpoint changes, and spatio-temporal uncertainty.

Vision-based vehicle
re-id approaches, including vehicle appearance, license plate, and spatio-temporal characteristics. 


re-identify the specific
vehicle that appeared in different cameras over the surveillance network. The vehicle re-id
module should recognize same vehicle that appears in surveillance cameras installed
in different geographical locations. Vehicle re-id can be treated as a fine-grained
recognition problem that identifies the subordinate type of input class.

Vehicle re-id can be of six categories
\begin{enumerate}
	\item Vision based Re-Id
	\item Magnetic Sensor Based Re-Id
	\item Inductance Based Re-Id
	\item GPS Based Re-Id
	\item Contextual Cues Based Re-Id
	\item Hybrid Methods Based Re-Id
\end{enumerate}


In computer vision, the aim of vehicle re-id is to identify specific vehicle that appeared
over in multiple cameras network. 
Five Basic steps
\begin{enumerate}
\item Data Collection
\item Bounding Box Generation obtained by vehicle detection technique.
\item Training Data Annotation
\item Model Training:  Learning discriminative
features and good values for all the weights and the bias from previous annotated vehicle videos or images of the dataset.
\item Vehicle Retrieval
\end{enumerate}

Eight different techniques have been employed in this research
area: 
(A) Feature representation for vehicle re-id, 

	Classified into two parts: hand-crafted and deep learning features representations. Handcrafted feature representation utilized in person
	re-id and then applied directly on vehicle re-id task. Deep learning based feature representations such as GoogLeNet [38], VGGNet [39], AlexNet [40], and,
ResNet [41] are used for vehicle re-id.
	
	Deep Joint Discriminative Learning (DJDL) [45] approach uses identification, and verification and
 triplet loss functions improved triplet convolutional neural network [46] uses classification and-oriented and triplet loss function to extract discriminative feature representation.
	
	
(B) Similarity metric for vehicle re-id,
	Distance metric learning approaches [56]
are thoroughly studied in image retrieval and recognition tasks, in which metric space is
	defined in such a way that features that belong to same class are kept closer and different are at distant
	
	As in various face recognition algorithms [57,58] uses Euclidean and
	Cosine distance metric to measure the similarity between the pair of vehicle for re-id.
	
	Furthermore,
deep relative distance learning (DRDL) [44] studied a two-branch convolutional neural
network to covert the raw vehicle images into a Euclidean space, so that distance can be
used directly to measure the similarity of two individual vehicles.
	

(C) Traditional machine learning-based vehicle re-id, 
	Extracted features are directly computed from image pixels and its low level feature representation.
	
	Algorithms proposed for low level feature extraction
		Speeded up robust features (SURF) [47],
		scale-invariant feature transform (SIFT) [48], 
		histogram of oriented gradient (HOG).
		
	After feature extraction different classifiers are applied, which are widely used in TML
approaches such as linear regression, k-Nearest Neighbor (KNN) [49], logistic regression,
support vector machine (SVM) [50], bayes classification [51], and decision tree [52]. 
	
	Zapletal and
Herout [53] utilize the color histogram and the HOG features with linear regression to
re-id vehicle. 
	Chen et al. [54] designed a method to re-id vehicles grid-by-grid with HOG
features extraction for coarse search and further improves the result by utilizing histograms
of matching pairs.
	
	
(D) View-aware-based vehicle re-id,

	Most of the above discussed deep learning features [38,39,45] are general, and these
learned features end at multiple fully connected layers. 
	
	Despite that, all these approaches
performance is not bad. But these approaches are not designed for a specific problem
related to view point variation. 
	
	Zhao et al. [64]
designed a novel approach based
on person body parts guided for re-id. Wu et al. [65] proposed a study with pose prior
	that made identification efficient and robust to viewpoint. Zheng et al. [66] proposed
the pose box structure that generates the pose estimation after affine transformations. Prokaj et al. [68] proposed a pose estimation-based approach to handle multiple
viewpoint problem.
	
	Yi Zhou et al. [69] studied uncertainty in the viewpoint of vehicle re-id
system and designed end to end deep learning-based architecture on Long Short-Term
Memory (LSTM) bi-directional loop and concatenated CNN, in this model author takes
full advantage of LSTM and CNN to learn the different viewpoints of vehicle. 
	
(E) Fine-grained visual recognition-based vehicle re-id, 
	Divided into two parts, representation learning model and part-based model. 
	
	Many
approaches are proposed [60] that utilize alignment and part localization for feature extraction of main parts and then those parts are compared for vehicle re-id. 
	
		Reinforcement learning to get
discriminative parts of vehicle.
		Bilinear architecture
to get the pair of local features
		Utilize shape and lights of vehicle visible in night 
		
		
	In fine-grained recognition, local region features are extracted from different points
such as logo, annual inspection stickers, and decorations, to make system more efficient
and robust various attributes of vehicles are also incorporated like color, model, and type
information. 
	
(F) Generative adversarial network-based vehicle re-id, 

(G) Attention mechanism, 
(H) License plate-based vehicle re-id.
