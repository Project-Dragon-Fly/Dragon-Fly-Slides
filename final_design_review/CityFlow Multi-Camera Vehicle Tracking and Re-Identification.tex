CityFlow: A City-Scale Benchmark for Multi-Target
Multi-Camera Vehicle Tracking and Re-Identification \cite{Tang_2019_CVPR}

Urban traffic optimization using traffic cameras as sensors is driving the need to advance state-of-the-art multitarget multi-camera (MTMC) tracking. 

To achieve this goal, one has to address three distinct but closely related research problems: 1) Detection and tracking of targets within a single camera, known as multi-target singlecamera (MTSC) tracking; 2) Re-identification of targets
across multiple cameras, known as ReID; and 3) Detection
and tracking of targets across a network of cameras, known
as multi-target multi-camera (MTMC) tracking. 


The two main challenges in vehicle
ReID are small inter-class variability and large intra-class
variability, i.e., the variety of shapes from different viewing
angles.

A major limitation of existing benchmarks for object
ReID is the limited spatial coverage and small number of cameras used.


Prev literature
	the cameras span less than 300 × 300 m2 ,
with only 6 and 8 views, respectively. 
	Do not provide the original videos or
camera calibration information. 
	Assumes that image signatures
are grouped by correct identities within each camera, which
is not reflective of real tracking systems. 
	Only the front and back views of the
vehicles are available, thus limiting the variability due to
viewpoint. 
	
	Most state-of-the-art approaches on these
benchmarks exploit metric learning to classify object identities, where common loss functions include hard triplet
loss [13], cross entropy loss [40], center loss [48], etc.
	
	On the other hand, the computation of deep learning features is costly, and thus spatio-temporal reasoning using
video-level information is key to applications in the real
world. 
	

In this paper
The first
benchmark at city scale for MTMC tracking
the nature of the synchronized high-quality videos,
the large spatial expanse captured by the
dataset. 

contains the largest number of cameras (40) from a large
number of intersections (10) 
covering a variety of scenes 
fully labeled, homography matrices that relate pixel
locations to GPS coordinates are available to enable precise spatial localization. 


It is important to leverage the spatio-temporal
information to address the city-scale problem properly.



For the person ReID problem, the state-of-the-art apply
metric learning with different loss functions, such as hard
triplet loss (Htri) [13], cross entropy loss (Xent) [40], center loss (Cent) [48], and their combination to train classifiers [62]. 

In our experiments, we compared the performance of various convolutional neural network (CNN) models [12, 54, 16, 51, 17, 38, 36], which are all trained
using the same learning rate (3e-4), number of epochs (60),
batch size (32), and optimizer (Adam). 

For the vehicle ReID problem, the recent work [18] ex-
plores the advances in batch-based sampling for triplet em-
bedding that are used for state-of-the-art in person ReID
solutions. They compared different sampling variants and
demonstrated state-of-the-art results on all vehicle ReID
benchmarks [28, 26, 52], outperforming multi-view-based
embedding and most spatio-temporal regularizations (see
Tab. 7). Chosen sampling variants include batch all (BA),
batch hard (BH), batch sample (BS) and batch weighted
(BW), adopted from [13, 35]. The implementation uses
MobileNetV1 [15] as the backbone neural network architec-
ture, setting the feature vector dimension to 128, the learn-
ing rate to 3e-4, and the batch size to 18 × 4.




5.2. MTSC tracking and object detection
Reliable cross-camera tracking is built upon accurate
tracking within each camera (MTSC). 

As for MTSC trackers, TC [43], the
only offline method, performs better according to most of the evaluation metrics. 


MTMC tracking is a joint process of visual-spatio-temporal reasoning. 
For these experiments, we first apply
MTSC tracking, then sample a number of signatures from
each trajectory in order to extract and compare appearance
 features. The number of sampled instances from each vehi-
cle is empirically chosen as 3. 

Note also that,
since only trajectories spanning multiple cameras are in-
cluded in the evaluation, different from MTSC tracking,
false positives are considered in the calculation of MTMC
tracking accuracy.


We can also conclude from Tab 9 that the
choice of image-based ReID and MTSC tracking methods
has a significant impact on overall performance, as those
methods achieving superior performance in their sub-tasks
also contribute to higher MTMC tracking accuracy.


We proposed a city-scale benchmark, CityFlow, which
enables both video-based MTMC tracking and image-based
ReID tasks. 

Our major contribution is three-fold.
	CityFlow is the first attempt towards city-scale applications
in traffic understanding. 
	
	CityFlow is also the first benchmark to
support vehicle-based MTMC tracking, by providing annotations for the original videos, the camera geometry, and
calibration information. The provided spatio-temporal information can be leveraged to resolve ambiguity in image based ReID.
	
	Third, we conducted extensive experiments
evaluating the performance of state-of-the-art approaches
on our benchmark, comparing and analyzing various visual-spatio-temporal association schemes. 
	